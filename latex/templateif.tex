\documentclass[a4paper,12pt]{article}
\usepackage[utf8]{inputenc}
\usepackage[T1]{fontenc}
\usepackage[brazil]{babel}
\usepackage{abntex2}
\usepackage{indentfirst}
\usepackage{setspace}
\usepackage{graphicx}
\usepackage{lipsum} % Pacote para gerar texto de exemplo. Pode ser removido.

% Configurações de margem conforme normas do IF Goiano
\setlength{\textwidth}{16cm}
\setlength{\textheight}{23cm}
\setlength{\topmargin}{-1cm}
\setlength{\oddsidemargin}{0cm}
\setlength{\evensidemargin}{0cm}

% Configuração de espaçamento 1,5 entre linhas conforme normas da ABNT
\onehalfspacing

% Configuração de recuo do parágrafo conforme normas da ABNT
\setlength{\parindent}{1.25cm}

% Informações do TCC
\title{Título do TCC}
\author{Seu Nome}
\date{Data de Conclusão}

\begin{document}

\maketitle

\begin{resumo}
Este é o resumo do seu trabalho, conforme as normas da ABNT. O resumo deve ser conciso e informativo, destacando os principais pontos do seu TCC.
\end{resumo}

\textbf{Palavras-chave}: Palavra-chave 1. Palavra-chave 2. Palavra-chave 3.

\section{Introdução}
\label{sec:introducao}

Esta é a seção de introdução do seu trabalho. Você pode começar descrevendo o contexto e a importância do tema escolhido. Além disso, apresente o objetivo do seu TCC.

\section{Revisão Bibliográfica}
\label{sec:revisao}

Nesta seção, faça uma revisão da literatura relacionada ao seu tema. Discuta os principais conceitos, teorias e trabalhos anteriores relevantes.

\section{Metodologia}
\label{sec:metodologia}

Descreva a metodologia que você utilizou para realizar o seu TCC. Explique os métodos de coleta de dados, análise e outras etapas relevantes.

\section{Resultados}
\label{sec:resultados}

Apresente os resultados da sua pesquisa de forma clara e objetiva. Utilize gráficos, tabelas e figuras, se necessário.

\section{Discussão}
\label{sec:discussao}

Nesta seção, discuta os resultados obtidos e relacione-os com a literatura revisada na seção de revisão bibliográfica.

\section{Conclusão}
\label{sec:conclusao}

Faça uma conclusão geral do seu trabalho, destacando os principais achados e suas implicações.

\section{Referências}
\label{sec:referencias}

Liste todas as fontes bibliográficas utilizadas no seu TCC de acordo com as normas da ABNT.

\end{document}
